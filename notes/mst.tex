\documentclass[11pt]{article}
\usepackage[margin=1in,footskip=0.25in]{geometry}

\usepackage{braket,amsfonts}
\usepackage[english]{babel}
\usepackage[utf8]{inputenc}
\usepackage{amsmath}
\usepackage{graphicx}
\usepackage{amssymb}
% \usepackage{amsthm}
\usepackage{mathrsfs}
\usepackage[colorinlistoftodos]{todonotes}
\usepackage{enumitem}
\usepackage{yfonts}
% \usepackage{algorithm}
% \usepackage[linesnumbered,ruled]{algorithm2e}
% \usepackage[noend]{algpseudocode}
\usepackage{algorithmic}
\usepackage{cite}

\newcommand{\dd}[1]{\mathrm{d}#1}
\usepackage{filecontents}
\usepackage{ulem}
% to order citations
% \usepackage{cite}
% to allow clickable references
% \usepackage[hidelinks]{hyperref}
% for \coloneqq
\usepackage{mathtools}
\usepackage{adjustbox}
\usepackage{lipsum}% example text
\usepackage[section]{placeins}
\DeclareMathOperator*{\vecn}{vec}
\usepackage{bm}
% \usepackage{subfig}
\usepackage{graphicx,epstopdf}% <- Preamble
\usepackage[caption=false]{subfig}% <- Preamble

\usepackage{listings}


\title{}

\author{}
\newtheorem{thm}{Theorem}[section]
\newtheorem{lem}[thm]{Lemma}

\newtheorem{defn}[thm]{Definition}
\newtheorem{eg}[thm]{Example}
\newtheorem{ex}[thm]{Exercise}
\newtheorem{conj}[thm]{Conjecture}
\newtheorem{cor}[thm]{Corollary}
\newtheorem{claim}[thm]{Claim}
\newtheorem{rmk}[thm]{Remark}

\newcommand{\ie}{\emph{i.e.} }
\newcommand{\cf}{\emph{cf.} }
\newcommand{\into}{\hookrightarrow}
\newcommand{\dirac}{\slashed{\partial}}
\newcommand{\R}{\mathbb{R}}
\newcommand{\C}{\mathbb{C}}
\newcommand{\Z}{\mathbb{Z}}
\newcommand{\N}{\mathbb{N}}
\newcommand{\LieT}{\mathfrak{t}}
\newcommand{\T}{T}
\newcommand{\tree}[1]{{\mathcal{#1}}}
\newcommand{\tsr}[1]{\pmb{\mathcal{#1}}}
\newcommand{\fvcr}[1]{\bm{#1}}
\newcommand{\vcr}[1]{\mathbf{#1}}
\newcommand{\mat}[1]{\mathbf{#1}}
%\newcommand{\defeq}{\coloneqq}
\newcommand{\defeq}{=}
%\newcommand{\tnrm}[1]{\lvert \lvert #1 \rvert \rvert_F}
\newcommand{\tnrm}[1]{{\| #1 \|}_2}
\newcommand{\fnrm}[1]{{\| #1 \|}_F}
%\newcommand{\tinf}[1]{\left|\sigma_\text{min}(#1)\right|}
\newcommand{\tinf}[1]{\inf\{\vnrm{\fvcr{f}_{#1}}\}}
%\newcommand{\vnrm}[1]{{\lvert \lvert #1 \rvert \rvert}_2}
\newcommand{\vnrm}[1]{{\| #1 \|}_2}
%\newcommand{\vnrm}[1]{\| #1 \|_2}
\newcommand{\inti}[2]{\{{#1},\ldots, {#2}\}}
\newcommand{\M}{M}
\newcommand\bigzero{\makebox(0,0){\text{\huge0}}}
\newcommand{\es}[1]{{\color{red} #1}}



\begin{document}
\maketitle

\section{First cut implementation}
\textbf{Edge relaxation/conditional hooking:} Let $p$ be a vector of ints and A be a matrix of EdgeExt (key, weight, parent).\\
Analgous to $p["i"] += A["i"] * p["j"]$ for hooking.

\begin{lstlisting}
Algorithm:
  while $p_prev != p$:
    $p_prev = p$
    $p["i"] += A["i"] * p["j"]$

    while $s_prev != p$
      $s_prev = p$
      $p[p[i]] = s_prev$
\end{lstlisting}

Note $+=$ ensures hooking on roots.\\

For current mst implementation, $p["i"] += A["i"] * p["j"]$ is computed with the following routine:
\begin{lstlisting}
    Edge Relaxation:
      auto q = new Vector<EdgeExt>(n, p->is_sparse, *world, MIN_EDGE);
      (*q)["i"] = Function<int,EdgeExt>([](int p){ return EdgeExt(INT_MAX, INT_MAX, p); })((*p)["i"]);
      Bivar_Function<EdgeExt,int,EdgeExt> fmv([](EdgeExt e, int p){ return EdgeExt(e.key, e.weight, p); });
      fmv.intersect_only=true;
      (*q)["i"] = fmv((*A)["ij"], (*p)["j"]);
      (*p)["i"] += Function<EdgeExt,int>([](EdgeExt e){ return e.parent; })((*q)["i"]);
\end{lstlisting}

For all $i$, this routine finds computes $minedge = \min_{j} A["ij"]$ and sets $p[i] = minedge.parent$. However, our current implementation fails to perform any hooking after the initial step. The nodes are ``happy'' with their minimum edge and do not want to take any other edge.\\

Potential fixes: ``zeroing out'' entries in $A$ to signify that the corresponding edge has already been taken, project matrix to $\#(stars) x \#(stars)$, or enlarge EdgeExt to have an additional boolean parameter $taken$.\\

We also must consider how to track mst, as $p$ will only contain information about the current forest (ie not where edges originally came from/were going to).

\section{Meeting Notes}
20-Feb:
\\
Comment after discussion: The other edges will still be taken (similar to connectivity). An example graph of 3 nodes where 0 is connected to 2 with edge weight 10, and 0 is also connected to 1 with edge weight 30. 0 would pick node 1 in the 1st iteration, and 1 would pick 3 ``through'' 0 in the next iteration.
The parent vector can also have a pseudo parent vector which keeps track of the edge each node initially took to get connected. This will essentially capture the MST.

\end{document}
